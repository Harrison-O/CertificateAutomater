\usepackage[toc, acronym]{glossaries}

\makeglossaries

\newacronym{ciet}{CIET}{Curtin IET On Campus}

\newacronym{iet}{IET}{Institute of Engineering and Technology}

\newacronym{stem}{STEM}{Science, Technology, Engineering, and Mathematics}

\newacronym{cpd}{CPD}{Continuing Professional Development}

\newacronym{pdf}{PDF}{Portable Document Format}

\newacronym{uri}{URI}{Uniform Resource Identifier}

\newacronym{url}{URL}{Uniform Resource Location}

\newacronym{api}{API}{Application Programming Interface}

\newacronym{cli}{CLI}{Command Line Interface}

\newacronym{csv}{CSV}{Comma Separated Values}

\newacronym{mvp}{MVP}{Minimal Viable Product}

\newacronym{gui}{GUI}{Graphical User Interface}

\newacronym{http}{HTTP}{Hypertext transfer protocol}

\newacronym{html}{HTML}{Hypertext Markup Language}

\newacronym{tcp}{TCP}{Transfer Control Protocol}

\newacronym{oop}{OOP}{Object Oriented Programming}

\newacronym{json}{JSON}{JavaScript Object Notation}

\newacronym{io}{IO}{Input/Output}

\newacronym{os}{OS}{Operating System}

\newacronym{vcs}{VCS}{Version Control Software}

\newglossaryentry{mailchimp}{
    name=Mailchimp,
    description={Online service for small businesses to send out modern emails to contacts}
}

\newglossaryentry{canva}{
    name=Canva,
    description={Simple webapp graphic design software for non-professionals}
}

\newglossaryentry{library}{
    name=library,
    description={Collection of theme consistent code to be reused in applications}
}

\newglossaryentry{poc}{
    name={proof of concept},
    description={A realisation of an idea to show its feasibility}
}

\newglossaryentry{path}{
    name=path,
    description={A sequence of folder names, optionally ending in a filename, to show that location of a resource}
}

\newglossaryentry{server}{
    name=server,
    description={A piece of computer hardware that provides functionality for other programs on other devices}
}

\newglossaryentry{client}{
    name=client,
    description={The machine that uses a server's resources on behalf of a user operating the machine}
}

\newglossaryentry{cla}{
    name={command line arguments},
    description={An argument passed when starting a command line program}
}

\newglossaryentry{batch}{
    name=batch,
    description={A collection of Mailchimp requests to be bundled and processed simultaneously}
}

\newglossaryentry{user}{
    name=user,
    description={Human actor interacting with program or system}
}

\newglossaryentry{class}{
    name=class,
    description={In programming, a template for objects}
}

\newglossaryentry{object}{
    name=object,
    description={In programming, a collection of data (usually mutatible) and functions that models a single entity}
}

\newglossaryentry{docstring}{
    name=docstring,
    description={A special comment in Python as the first line of a constant, class, or function with triple quotes.}
}

\newglossaryentry{python}{
    name=Python,
    description={Modern programming language, interpretated and weak-typed with simply syntax and focus on scientific and prototyping purposes}
}

\newglossaryentry{webapp}{
    name=Webapp,
    description={An application that runs in a browser}
}

\newglossaryentry{browser}{
    name=browser,
    description={An application that allows a user and a client machine to browse the internet}
}

\newglossaryentry{log}{
    name=log,
    description={A message recording an action that a program took}
}

\newglossaryentry{identification}{
    name=identification,
    description={Providing a system of actor one's own identity}
}

\newglossaryentry{authentification}{
    name=authentification,
    description={The process of confirming or denying the provided identity}
}

\newglossaryentry{authorisation}{
    name=Authorisation,
    description={The process of granting rights or privileges based on the authenticated identity}
}

\newglossaryentry{sphinx}{
    name=Sphinx,
    description={In Python programming, a toolset for creating heavilyt automated technical documentation of a codebase}
}

\newglossaryentry{pseudocode}{
    name=pseudocode,
    description={A language halfway between source code and English, used to convey an algorithm without being biased towards any specific language}
}

\newglossaryentry{src-code}{
    name={source code},
    description={The code that a programmer writes that can run on a computer}
}

\newglossaryentry{request}{
    name=request,
    description={In HTTP, when a client sends a message requesting access to a resource}
}

\newglossaryentry{response}{
    name=response,
    description={In HTTP, the message the server responds with to a client's request}
}

\newglossaryentry{metadata}{
    name=metadata,
    description={Data about the data, such as timestamps, locations, user operating, network info, etc.}
}

\newglossaryentry{encapsulation}{
    name=encapsulation,
    description={In programming, an OOP principle for hiding and protecting data}
}

\newglossaryentry{abstraction}{
    name=abstraction,
    description={In programming, and OOP principle for hiding the implementation of functionality}
}

\newglossaryentry{inheritance}{
    name=inheritance,
    description={In programming, an OOP principle for allowing one class to use the functionality of another class}
}

\newglossaryentry{polymorphism}{
    name=polymorphism,
    description={In programming, an OOP principle for allowing one class to overload or override a function}
}

\newglossaryentry{dunder}{
    name=dunder,
    description={double underscore}
}

\newglossaryentry{association}{
    name=association,
    description={In programming, an OOP tactic where one object is containted inside another}
}

\newglossaryentry{array}{
    name=array,
    description={In programming, a collection of items of the same datatype stored contiguous in memory}
}

\newglossaryentry{token}{
    name=token,
    description={In programming, a computer generated string used to authenticate a client to a server}
}

\newglossaryentry{blocking-func}{
    name={blocking function},
    description={In programming, a function that halts a program for a long time until a task completes}
}

\newglossaryentry{cython}{
    name=Cython,
    description={A superset of Python that allows for static typing and calling C code}
}

\newglossaryentry{framework}{
    name=framework,
    description={A set of prepackaged tools for solving common development problems}
}

\newglossaryentry{venv}{
    name={virtual environment},
    description={An isolated copy of a programming environment on a device, independent of the rest of the device}
}

\newglossaryentry{make}{
    name=makefile,
    description={A program for recompiling a codebase easily}
}
